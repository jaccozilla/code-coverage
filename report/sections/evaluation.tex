The evaluation of this senior project was done in two parts: first, an evaluation of the Multi-file Code Coverage Tool; and second, an evaluation of developing it for DrRacket.

The Multi-file Code Coverage tool successfully extends DrRacket's code coverage to multiple source files. It does so without requiring any modifications to DrRacket's source code. Additionally, it provides concrete information to the user on the amount of covered code versus uncovered code. These were the main foci of this senior project. The Multi-file Code Coverage tool has been available on PLaneT for, at the time of this writing, approximately two weeks. Thus far, no bugs have been reported.

While the tool meets all of the requirements, it does have a few areas that could be improved upon. First, the detection of valid files does not work in some cases. Specifically, files that have been modified, but not saved, and are not in the focused \emph{tab} of a \emph{frame}, will report that the coverage is valid. Fixing this would require learning more details of the DrRacket user interface and its management. Second, the Multi-file Code Coverage Tool could be slightly better integrated with DrRacket. One possible way of doing this would be extending the ``Run'' button to automatically save coverage information. Currently, if the user runs the program, modifies it, and then attempts to load multi-file coverage, it will load outdated data. By also integrating the tool with the ``Run'' button the most recent test coverage information would always be saved. Finally, the parsing of the test coverage info hash table is not particularly fast, especially on larger projects. Further research could be done to improve the efficiency of the algorithm, cache data, or some other method to speed it up.

Overall, developing the Multi-file Code Coverage Tool for DrRacket went smoothly. DrRacket is well documented and there are many tutorials available on docs.racket-lang.org. However, there were a few issues that were encountered during the development of the tool: one, lower level methods and variables were not as well documented; two, finding new information was sometimes difficult; and three, there were limited external Racket resources. All of the issues presented next may actually have solutions that were not discovered. However, if this is the case then another issue is present: understanding DrRacket is difficult and takes a long time. Each issue presented was exhaustively researched by myself, and if I failed to find the correct answer then better documentation and explanations are needed.

While most high level, and often used, functions are well documented, lower level ones are not as well. This presented challenges when trying to determine what the purpose of the function was. One specific example is the variable called \emph{test-coverage-enabled}. There are actually two separate variables in DrRacket's source, both named \emph{test-coverage-enabled}. One used by DrRacket's user interface and one used by the code coverage generation. Additional documentation could have made this clearer. Overall DrRacket's low level documentation was much better than what I have seen in other projects.

Finding related functions and documentations was not always obvious. One example of this was seen when attempting to find all of the active DrRacket \emph{frames}. I thought I had searched everywhere for a function to find them. It was not until I was pointed to the \emph{group:get-the-frame-group} function that I was able to find the needed functions. No where else had I seen mentions of groups and had no reason to search for them. However, this is not a problem inherent with DrRacket, but one that is present in projects with a large code base.

Finally, since Racket and DrRacket are not as widely used as Java or C++, there were limited external resources on the subject. This presented problems when I would attempt to research, what I felt would be common error messages. Often I would only find one or two results, both on the Racket mailing list. The answers provided there would not always fix my problem. This is in comparison to searching a Java error, where pages of results, that include sites such as Stack Overflow, are displayed. This forced me to ask my advising professor, John Clements, many questions that had simple answers I could have figured out independently had it been a more common language. 