The evaluation of this senior project will be done in two parts: first, an evaluation of the Multi-file Code Coverage Tool; and second, an evaluation of developing for DrRacket.

The Multi-file Code Coverage tool successfully extends DrRacket's code coverage to multiple source files. It does so without requiring any modifications to DrRacket's source. Additionally, it provides concrete information to the user on the amount of covered code versus uncovered code. These were the main focuses of this senior project. The Multi-file Code Coverage tool has been available on planet for, at the time of this writing, approximately two weeks. Thus far no bugs have been reported.

While the tool meets all of the requirements, it does have a few areas that could be improved. First, the detection of valid files does not work in some cases. Specifically, files that have been modified, but not saved, and are not in the focused \emph{tab} of a \emph{frame} will report that the coverage is valid. Fixing this would require learning more details of the DrRacket user interface and its management. Second, the Multi-file Code Coverage Tool could be slightly better integrated with DrRacket. One possible way of doing this would be extending the ``Run'' button to automatically save coverage information. Currently, if the user runs the program, modifies it, and then attempts to load multi-file coverage it will load outdated data. By integrating with the ``Run'' button the most recent test coverage information would always be saved. Finally, the parsing of the test coverage info hash table is not particularly fast, especially on larger projects. Further research could be done to improve the efficiency of the algorithm, cache data, or some other method to speed it up.