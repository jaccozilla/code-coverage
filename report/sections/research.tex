Before attempting to implement the Multi-file Code Coverage Tool research was required to better understand how DrRacket implements and uses code coverage. This research consisted primarily of modifying small sections of DrRacket and then examining the results. While this process was not particularly fast, it did eventually reveal the information needed to successfully implement the tool.

During the research processes a few important facts where discovered: one,organization of DrRacket's user interface; the two, DrRacket's test info hash table contains information for all uncompiled required files; three, code coloring is applied independently from the code coverage collection process; and four, the test info hash table persists after code coloring has been applied.

DrRacket's user interface is composed of four main classes. The highest class is the \emph{frame}. The \emph{frame} is an entire DrRacket environment as seen in figure \ref{fig:drracket}. Each frame has at least one \emph{tab}. A \emph{tab} always contains a definitions window, where the source code is visible and editable. A \emph{tab} may also contain an interactions window. Additionally, the frame group contains a list of all currently open \emph{frames}.

\newfigure{gui-classes}{User Interface Classes}{width=6cm}{h!}

While DrRacket's default code coverage only applies code coloring to the active file, it collects information for all uncompiled files in the project. Code coverage can not be collected for compiled files because their code can not be expanded to track executed expressions (see figure~\ref{fig:default-flow}). 

The code coverage and code coloring process are independent. DrRacket generates a hash table, called \emph{test-coverage-info} internally, when the program ins run. The hash table stores expressions, with their location in the source code, and whether or not they have been evaluated. 

The test coverage info hash table can be sent to a method in \emph{tab} (\ref{fig:gui-classes} which does code coloring for that tab based on the hash table. This hash remains available for use after code coloring has been completed. 

